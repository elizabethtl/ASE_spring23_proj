\section{Bonus: Ablation Study}
\label{sec:bonus_ablation}


\begin{table}[]
  \begin{tabular}{lllll}
                      & CityMPG+ & HighwayMPG+ & Weight- & Class- \\
                      \hline
  none                & 26.02    & 32.38       & 2661.03 & 13.75  \\
  luggage             & 25.69    & 32.05       & 2702.21 & 14.92  \\
  manual transmission & 25.22    & 31.82       & 2689.13 & 14.23  \\
  maker               & 25.11    & 31.53       & 2812.98 & 16.59 
  \end{tabular}
  \caption{removed features and model performance in ablation study}
  \label{tab:ablation}
  \end{table}


  For our ablation study, we used the auto2 dataset. We removed one of
  the x columns each time and ran sway on the new data. This would help
  us in identifying which features are more important than others. We
  listed the features that had made a noticeable difference in the
  objective values when removed in table \ref{tab:ablation}. Features
  that were not listed did not yield noticeable difference when we ran
  our experiments. As shown in the table, we can infer that three of the
  features: the maker of the car, the luggage capacity, and whether
  manual transmission was available were important in maximizing or
  minimizing the objectives.