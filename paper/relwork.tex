\section{Related Work}
\label{sec:relwork}
The algorithm used in this paper, Sway~\cite{chen2018sampling}, is a
method for search-based software engineering.
Search based software engineering (SBSE) was first proposed by Harman and
Jones~\cite{harman2001search} in 2001. SBSE transforms a software
engineering problem to a search problem to apply metaheuristic search.
A software engineering problem can be reformed as a search problem by
defining the following: a representation of the problem, a fitness
function, and a set of manipulation operators. Common algorithms include
random search, simulated annealing, genetic algorithms.

The advantage of metaheuristic algorithms is that they can explore
multiple objectives simultaneously. Multiple-objective evolutionary
algorithms (MOEA) are used in SBSE to help achieve multi-objective
optimization (MOO). In multi-objective problems, there usually isn't a
single optimal solution~\cite{marler2004MOOsurvey}, a set of optimal
points, such as the pareto frontier, is determined. 


\subsection{Random Projection}
% https://en.wikipedia.org/wiki/Random_projection

\subsection{Semi-Supervised Learning}
% https://www.molgen.mpg.de/3659531/MITPress--SemiSupervised-Learning.pdf

\subsection{Why Heuristics Work}
% http://library.mpib-berlin.mpg.de/ft/gg/gg_why_2008.pdf