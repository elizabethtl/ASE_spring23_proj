\section{Methods}
\label{sec:method}

Sway recursively splits the dataset in half and finds the best cluster
using a \textit{split} function. The \textit{split} function picks a
random point and the two points with the largest Euclidean distance from
it. The two furthest points form a line, which split all other points
into two halves by calculating the distance of the x columns (the
columns that are not objectives). Then one point from each of the two
halves are compared through the \textit{better} function, to determine
which point has better objectives (y columns). The \textit{split}
function is then executed on the better half. This process repeats until
we reach a certain cluster size, by then we would have the best cluster.


Our new proposed method built on top of sway is to add randomness when
splitting data into two halves. The \textit{half()} splits data into two
halves in sway, it first sorts all data according to the x columns.
These x columns provide a basis for us to cluster similar data together.
We do not compare the y columns when splitting data since we assume that
in real-world scenarios it would be costly to acquire such values.
However, we do don't know the
exact relation between the x columns (characteristics) and the y columns
(objectives). Thus, we propose adding randomness, so we don't fully rely
on unknown patterns in the data. We hope that the introduced randomness
will help us find better objective values. In the following sections, we
refer to the original sway as sway1, and sway2 as our new method.

We next describe the main implementation of randomness in our code. The
\verb|dist| function in the data class calculates the distance of the
x columns for two rows. We use a random coefficient (\verb|RAND| in the
pseudocode below), to vary the distance by a certain percentage. In our
experiments, we set this percentage to 15\%. This means a distance of 1
would return a value ranging from -0.85 to 1.15.

\begin{lstlisting}
  def dist(row1, row2):
    n = 0
    d = 0 
    for col in x_columns:
      n += 1
      d += col.dist(row1[col_index], 
              row2[col_index])^constant
      d = d * (1+ random_num(-RAND, RAND))
    return (d/n)**(1/dist_coefficient)
\end{lstlisting}

We then applied sway to the \textit{xpln} method. The \textit{xpln}
method takes the best cluster from sway and another random sample of
data and tries to find a rule that would most effectively distinguish
between the two. Xpln1 would be sway1 applied to xpln, and xpln2 is
sway2 applied to xpln.  

The \textit{top} method goes through all data in the dataset to compare
and sort based on a \textit{better} metric. The \textit{better} metric
compares the y columns of the data. As we compared each pair of data
points, this resulted in $O(n^2)$ time. As some datasets had a large number
of rows, we did not run the \textit{top} method for certain datasets due
to time concerns. 


\begin{table*}[h]
  \begin{center}
    

% Please add the following required packages to your document preamble:
% \usepackage{multirow}

  \begin{tabular}{llllll}
  \hline
  dataset                & characteristic    & mean     & median  & mode  & standard deviation \\
  \hline\hline
  auto2                  & CityMPG+          & 22.37    & 21      & 18    & 5.62               \\
                         & HighwayMPG+       & 29.09    & 28      & 26    & 5.33               \\
                         & Weight-           & 3072.9   & 3040    & 3470  & 589.9              \\
                         & Class-            & 19.51    & 17.7    & 15.9  & 9.69               \\
  \hline
  auto93                 & Lbs-              & 2970.42  & 2803.5  & 2130  & 846.84             \\
                         & Acc+              & 15.57    & 15.5    & 14.5  & 2.76               \\
                         & Mpg+              & 23.84    & 20      & 20    & 8.34               \\
  \hline
  china                  & N\_effort-        & 4277.64  & 2098    & 296   & 7071               \\
  \hline
  coc1000                & LOC+              & 1013.05  & 1060.5  & 720   & 571.35             \\
                         & AEXP-             & 2.97     & 3       & 2     & 1.2                \\
                         & RISK-             & 6.68     & 5       & 0     & 6.37               \\
                         & EFFORT-           & 30807.5  & 19642   & 33906 & 33883.81           \\
  \hline
  coc10000               & Loc+              & 1009.04  & 1012    & 100   & 574.75             \\
                         & Risk-             & 6.59     & 5       & 0     & 6.04               \\
                         & Effort-           & 30506.37 & 19697.5 & 4509  & 35435.43           \\
  \hline
  health...0001-hard     & MRE-              & 82.32    & 75.04   & 199   & 12.45              \\
                         & ACC+              & 5.15     & 7.14    & 0     & 3.82               \\
                         & PRED40+           & 22.1     & 25      & 25    & 13.52              \\
  \hline
  health...0011-easy     & MRE-              & 92.3     & 119.33  & 0     & 48.43              \\
                         & ACC+              & -8.53    & -12.24  & 0     & 5.71               \\
                         & PRED40+           & 17.79    & 0       & 0     & 34.13              \\
  \hline
  nasa93dem              & Kloc+             & 94.02    & 47.5    & 100   & 133.6              \\
                         & Effort-           & 624.41   & 252     & 60    & 1135.93            \\
                         & Defects-          & 3761.76  & 2007    & 2077  & 6145.06            \\
                         & Months-           & 24.18    & 21.4    & 13.6  & 12.97              \\
  \hline
  pom                    & Cost-             & 369.99   & 327.32  & 0     & 204.40             \\
                         & Completion+       & 0.87     & 0.9     & 1     & 0.13               \\
                         & Idle-             & 0.24     & 0.23    & 0     & 0.2                \\
  \hline
  SSM                    & NUMBERITERATIONS- & 30.94    & 7       & 5     & 94.53              \\
  \hline
  SSN                    & PSNR-             & 44.53    & 45.91   & 45.98 & 6.47               \\
                         & Energy-           & 1658     & 1258.09 & 0     & 1610.66           
  \end{tabular}

  \end{center}
  \caption{Dataset statistics}
  \label{tab:dataset}
\end{table*}

\subsection{Data}
  Our data consists of ten datasets, each has two types of columns.
  Columns with string values and columns with numeric values. Some
  columns have a plus or minus sign, this means they are the objectives,
  a minus sign means we try to minimize this value, while a plus sign
  means we try to maximize the value. Table \ref{tab:dataset} shows
  statistics on these objective columns, which we use as a baseline to
  compare the results from our experiments.

  To understand the data, let us take auto2 as an example, the auto2
  dataset has a total of twenty-three columns,
  which includes four objectives, or y columns: \textit{CityMPG+,
  HighwayMPG+, Weight-, Class-}. The remaining nineteen columns,
  \textit{type, engine\_size, horsepower, etc.}, are x columns. 
  Our methods would cluster data based on the nineteen x columns and
  choose data from each cluster to most effectively
  maximize or minimize the values of the objectives.


\subsection{Experiments}
  We ran our model on samples of size 10, 25, 50, 100, 200, 500, and
  1000. For each sample size, we ran 20 repeated runs with different
  seeds and calculated the average of the values we collected. The
  source code and experiment outputs can be found in the GitHub
  repository. 



  