%% bare_conf.tex
%% V1.4b
%% 2015/08/26
%% by Michael Shell
%% See:
%% http://www.michaelshell.org/
%% for current contact information.

\documentclass[conference]{IEEEtran}

\usepackage{xcolor}
\newcommand\myred[1]{\textcolor{red}{#1}}
\usepackage{listings}

% correct bad hyphenation here
\hyphenation{op-tical net-works semi-conduc-tor}


\begin{document}
%
% paper title
% Titles are generally capitalized except for words such as a, an, and, as,
% at, but, by, for, in, nor, of, on, or, the, to and up, which are usually
% not capitalized unless they are the first or last word of the title.
% Linebreaks \\ can be used within to get better formatting as desired.
% Do not put math or special symbols in the title.
\title{Bare Demo of IEEEtran.cls\\ for IEEE Conferences}


% author names and affiliations
% use a multiple column layout for up to three different
% affiliations
\author{\IEEEauthorblockN{Elizabeth Lin}
\IEEEauthorblockA{Department of Computer Science\\
North Carolina State University\\
Raleigh, North Carolina\\
Email: etlin@ncsu.edu}}



% make the title area
\maketitle

% As a general rule, do not put math, special symbols or citations
% in the abstract
\begin{abstract}
The abstract goes here.
\end{abstract}

% no keywords




% For peer review papers, you can put extra information on the cover
% page as needed:
% \ifCLASSOPTIONpeerreview
% \begin{center} \bfseries EDICS Category: 3-BBND \end{center}
% \fi
%
% For peerreview papers, this IEEEtran command inserts a page break and
% creates the second title. It will be ignored for other modes.
\IEEEpeerreviewmaketitle


\section{Introduction}
\label{sec:intro}

% no \IEEEPARstart
This demo file is intended to serve as a ``starter file''
for IEEE conference papers produced under \LaTeX\ using
IEEEtran.cls version 1.8b and later.
% You must have at least 2 lines in the paragraph with the drop letter
% (should never be an issue)
I wish you the best of success.

\hfill mds
 
\hfill August 26, 2015

\subsection{Subsection Heading Here}
Subsection text here.


\subsubsection{Subsubsection Heading Here}
Subsubsection text here.

\section{Related Work}
\label{sec:relwork}
The algorithm used in this paper, Sway~\cite{chen2018sampling}, is a
method for search-based software engineering.
Search based software engineering (SBSE) was first proposed by Harman and
Jones~\cite{harman2001search} in 2001. SBSE transforms a software
engineering problem to a search problem to apply metaheuristic search.
A software engineering problem can be reformed as a search problem by
defining the following: a representation of the problem, a fitness
function, and a set of manipulation operators. Common algorithms include
random search, simulated annealing, genetic algorithms.

The advantage of metaheuristic algorithms is that they can explore
multiple objectives simultaneously. Multiple-objective evolutionary
algorithms (MOEA) are used in SBSE to help achieve multi-objective
optimization (MOO). In multi-objective problems, there usually isn't a
single optimal solution~\cite{marler2004MOOsurvey}, a set of optimal
points, such as the pareto frontier, is determined. 


\subsection{Random Projection}
% https://en.wikipedia.org/wiki/Random_projection

\subsection{Semi-Supervised Learning}
% https://www.molgen.mpg.de/3659531/MITPress--SemiSupervised-Learning.pdf

\subsection{Why Heuristics Work}
% http://library.mpib-berlin.mpg.de/ft/gg/gg_why_2008.pdf

\section{Methods}
\label{sec:method}



Sway recursively splits the dataset in half and finds the best cluster
using a \textit{split} function. The \textit{split} function picks a
random point and the two points with the largest Euclidean distance from
it. The two furthest points form a line, which split all other points
into two halves by calculating the distance of the x columns (the
columns that are not objectives). Then one point from each of the two
halves are compared through the \textit{better} function, to determine
which point has better objectives (y columns). The \textit{split}
function is then executed on the better half. This process repeats until
we reach a certain cluster size, by then we would have the best cluster.


Our new proposed method is to add randomness when splitting data into two
halves. The \textit{half()} splits data into two halves in sway, it
first sorts all data according to the x columns. These x columns provide
a basis for us to cluster similar data together. We do not compare the y
columns when splitting data since we assume that in real-world scenarios
it would be costly to acquire such values. However, we do don't know the
exact relation between the x columns (characteristics) and the y columns
(objectives). Thus, we propose adding randomness, so we don't fully rely
on unknown patterns in the data. We hope that the introduced randomness
will help us find better objective values. Sway1 is the original sway,
and sway2 is where we added randomness to the distance.

We next describe the main implementation of randomness in our code. The
\textit{dist} function in the data class calculates the distance of the
x columns for two rows. We use a random coefficient (RAND in the
pseudocode below), to vary the distance by a certain percentage. In our
experiments, we set this percentage to 15\%.

\begin{lstlisting}
  def dist(row1, row2):
    n = 0
    d = 0 
  for col in x_columns:
    n += 1
    d += col.dist(row1[col_index], 
            row2[col_index])**dist_coefficient
  ## add randomness to comparing cols.x
  d = d * (1+ random_num(-RAND, RAND))

  return (d/n)**(1/dist_coefficient)
\end{lstlisting}

We then applied sway to the \textit{xpln} method. The \textit{xpln}
method takes the best cluster from sway and another random sample of
data and tries to find a rule that would most effectively distinguish
between the two. Xpln1 would be sway1 applied to sway, and xpln2 is
sway2 applied to xpln.  

The \textit{top} method goes through all data in the dataset to compare
and sort based on a \textit{better} metric. The \textit{better} metric
compares the y columns of the data. As some datasets had a large number
of rows, we did not run the \textit{top} method for certain datasets due
to time concerns.


\begin{table*}[h]
  \begin{center}
    

% Please add the following required packages to your document preamble:
% \usepackage{multirow}

  \begin{tabular}{llllll}
  \hline
  dataset                & characteristic    & mean     & median  & mode  & standard deviation \\
  \hline\hline
  auto2                  & CityMPG+          & 22.37    & 21      & 18    & 5.62               \\
                         & HighwayMPG+       & 29.09    & 28      & 26    & 5.33               \\
                         & Weight-           & 3072.9   & 3040    & 3470  & 589.9              \\
                         & Class-            & 19.51    & 17.7    & 15.9  & 9.69               \\
  \hline
  auto93                 & Lbs-              & 2970.42  & 2803.5  & 2130  & 846.84             \\
                         & Acc+              & 15.57    & 15.5    & 14.5  & 2.76               \\
                         & Mpg+              & 23.84    & 20      & 20    & 8.34               \\
  \hline
  china                  & N\_effort-        & 4277.64  & 2098    & 296   & 7071               \\
  \hline
  coc1000                & LOC+              & 1013.05  & 1060.5  & 720   & 571.35             \\
                         & AEXP-             & 2.97     & 3       & 2     & 1.2                \\
                         & RISK-             & 6.68     & 5       & 0     & 6.37               \\
                         & EFFORT-           & 30807.5  & 19642   & 33906 & 33883.81           \\
  \hline
  coc10000               & Loc+              & 1009.04  & 1012    & 100   & 574.75             \\
                         & Risk-             & 6.59     & 5       & 0     & 6.04               \\
                         & Effort-           & 30506.37 & 19697.5 & 4509  & 35435.43           \\
  \hline
  health...0001-hard     & MRE-              & 82.32    & 75.04   & 199   & 12.45              \\
                         & ACC+              & 5.15     & 7.14    & 0     & 3.82               \\
                         & PRED40+           & 22.1     & 25      & 25    & 13.52              \\
  \hline
  health...0011-easy     & MRE-              & 92.3     & 119.33  & 0     & 48.43              \\
                         & ACC+              & -8.53    & -12.24  & 0     & 5.71               \\
                         & PRED40+           & 17.79    & 0       & 0     & 34.13              \\
  \hline
  nasa93dem              & Kloc+             & 94.02    & 47.5    & 100   & 133.6              \\
                         & Effort-           & 624.41   & 252     & 60    & 1135.93            \\
                         & Defects-          & 3761.76  & 2007    & 2077  & 6145.06            \\
                         & Months-           & 24.18    & 21.4    & 13.6  & 12.97              \\
  \hline
  pom                    & Cost-             & 369.99   & 327.32  & 0     & 204.40             \\
                         & Completion+       & 0.87     & 0.9     & 1     & 0.13               \\
                         & Idle-             & 0.24     & 0.23    & 0     & 0.2                \\
  \hline
  SSM                    & NUMBERITERATIONS- & 30.94    & 7       & 5     & 94.53              \\
  \hline
  SSN                    & PSNR-             & 44.53    & 45.91   & 45.98 & 6.47               \\
                         & Energy-           & 1658     & 1258.09 & 0     & 1610.66           
  \end{tabular}

  \end{center}
  \caption{Dataset statistics}
  \label{tab:dataset}
\end{table*}

\subsection{Data}
  Our dataset consists of ten datasets, each has two types of columns.
  Columns with string values and columns with numeric values. Some
  columns have a plus or minus sign, this means they are the objectives,
  a minus sign means we try to minimize this value, while a plus sign
  means we try to maximize the value. Table \ref{tab:dataset} shows
  statistics on these objective columns.

\subsection{Experiments}
  We ran our model on samples of size 10, 25, 50, 100, 200, 500, and
  1000. For each sample size, we ran 20 repeated runs with different
  seeds and calculated the average of the values we collected.



  

\section{Results}
\label{sec:results}

In this section, we discuss results of our experiments. We specifically
discuss our
results from a sample size of 500 out of the many sample sizes. 
Table \ref{tab:auto2}-\ref{tab:coc10000} shows some of our results. For
more results, please refer to the GitHub repository.

\subsection{Sway1 vs. Sway2}
  We
  observe that adding randomness in our model achieved mixed results. In
  most cases, sway2 performs slightly worse or similarly to sway1.
  However, there are certain cases where sway2 performs better. For
  example, sway2 performs better when we try to maximize \textit{Kloc} for
  the \textit{nasa93dem} dataset. We indicate objectives where sway2
  performs better than as red in our tables. As sway2 introduces
  randomness, we can infer that 
  for these objectives, closely clustered data points based on the x
  columns do not necessarily relate to similar values for the objectives.
  For example, in table \ref{tab:coc1000}, closely clustered x columns
  for coc1000 may not
  have similar values for the \textit{Effort} objective. 
  The same can be seen in table \ref{tab:nasa93dem} and
  \ref{tab:coc10000}.
  Results labeled top are results yielded from comparing all data.
  Logically, this returned the best results, however it is also the most
  time-consuming method. Comparing sway to top, we observe that there is
  a noticeable decrease in performance, this is often referred to as the
  \textit{sampling tax}. This shows we sacrifice the optimal values for
  in exchange for time when implementing sway. 
  
\subsection{Xpln}
  We applied sway1 and sway2 to xpln, which we labeled as xpln1 and
  xpln2 in our tables. As shown from the tables, for most cases, xpln
  performs slightly worse than sway along. This loss in performance is
  the \textit{explanation tax}. This is due to the fact that xpln uses
  simpler rules to distinguish the data points. However, there are cases
  in which xpln performs better than sway, we infer that this shows that
  data for certain features can be clustered using simpler rules. We
  highlighted cases where xpln performs better than sway in blue in
  tables \ref{tab:auto2}, \ref{tab:coc1000}, \ref{tab:coc10000}.
  Understanding phenomenon like this could be useful where we would
  conduct future studies, as we discuss in section \ref{sec:bonus_february}.

\subsection{Sample size}
  We also conducted experiments with different sample sizes in an attempt
  to achieve better results with more samples. However, we did not observe
  a clear correlation between sample sizes and the objectives. Table
  \ref{tab:sample_nasa93dem} shows varying sample sizes for the
  \textit{nasa93dem} dataset. Upon receiving these results, we went back
  to check how sample size was implemented in the code. Sample size has an
  effect when selecting the two points when splitting data into two
  halves. The first point is selected at random, and the second point is
  selected from a group of data that is of sample size. However, that is
  the extent of which sample size has an effect. Thus, we infer that
  sample size does not have enough effect in our code to influence the
  results. 

\subsection{Effect size and Significance tests}
  With results from different models, we conducted the effect size test
  using cliff's delta and the significance test using bootstrap. The
  effect size test tests for relationship between two sets of data. We set
  the threshold to 0.4 for medium effects. The confidence for the
  significance test is set to 0.05. The conjunction of the two test
  between different results is shown in table \ref{tab:tests}. For those
  where the conjuction is \textit{=}, this means both tests return true,
  which we can infer to meaning the two results are similar. We observe
  that results from sway1 is similar to results from sway2. However, both
  results from sway are not similar to results from top. It can also be
  seen from the table that xpln returns slightly different results from
  sway. 

\begin{table}[]
  \begin{center}
  \begin{tabular}{lllll}
          & CityMPG+      & Class-     & HighwayMPG+ & Weight-   \\
    \hline
    sway1 & 26.02         & 13.75      & 32.38       & 2661.03 \\
    xpln1 & \myblue{27.07}   & 16.17      & 33.14       & 2632.12   \\
    sway2 & 24.363        & 16.86      & 31.25       & 2845.97   \\
    xpln2 & \myblue{25.26}  & 16.72      & 25          & 2781.65   \\
    top   & 37.16         & 9.26       & 41.75       & 2040.98    
  \end{tabular}
  \end{center}
  \caption{Results for auto2.csv}
  \label{tab:auto2}
\end{table}


\begin{table}[]
  \begin{center}
  \begin{tabular}{llll}
        & Lbs-    & Acc+  & Mpg+  \\
  \hline
  sway1 & 2240.94 & 16.80 & 29.51 \\
  xpln1 & 2416.26 & 15.44 & 26.40 \\
  sway2 & 2319.56 & 16.62 & 29.72 \\
  xpln2 & 2456.19 & 14.49 & 23.57 \\
  top   & 1998.07 & 19.77 & 40.76
  \end{tabular}
\end{center}
  \caption{Results for auto93.csv}
  \label{tab:auto93}
\end{table}


% \begin{table}[]
%   \begin{center}
%   \begin{tabular}{ll}
%         & N\_effort- \\
%   \hline
%   sway1 & 1800.38    \\
%   xpln1 & 2619.90    \\
%   sway2 & 1965.81    \\
%   xpln2 & 2431.86    \\
%   top   & 145.94    
%   \end{tabular}
%   \end{center}
%   \caption{Results for china.csv}
%   \label{tab:china}
% \end{table}


\begin{table}[]
  \begin{center}
  \begin{tabular}{llllll}
        & LOC+    & AEXP- & PLEX- & RISK- & Effort-  \\
  \hline
  sway1 & 1027.74 & 3.05  & 2.88  & 5.08  & 29321.05 \\
  xpln1 & 913.11  & \myblue{2.67}  & \myblue{2.73}  & 5.7 & 27416.45 \\
  sway2 & 999.59  & \myred{2.92}  & 3.02  & 6.12  & \myred{28323.93} \\
  xpln2 & 918.43  & \myred{2.66}  & \myblue{2.73}  & 6.0   & \myred{26910.59} \\
  top   & 1571.43 & 1.62  & 1.39  & 4.7   & 35116.16
  \end{tabular}
\end{center}
\caption{Results for coc1000.csv}
\label{tab:coc1000}
  \end{table}

\begin{table}[]
  \begin{center}
    \begin{tabular}{lllll}
          & Kloc+ & Effort- & Defects- & Months-   \\
    \hline
    sway1 & 70.52 & 428.92  & 2811.09  & 20.86 \\
    xpln1 & 72.85 & 450.05  & 2894.04  & 21.40     \\
    sway2 & \myred{85.92} & 524.90  & 3289.46  & 22.04     \\
    xpln2 & \myred{84.24} & 542.82  & 3374.83  & 22.59     \\
    top   & 4.59  & 18.29   & 143.51   & 8.24     
    \end{tabular}
  \end{center}
  \caption{Results for nasa93dem.csv}
  \label{tab:nasa93dem}
\end{table}

\begin{table}[]
  \begin{center}
    \begin{tabular}{lllll}
          & MRE- & ACC+ & PRED40+   \\
    \hline
    sway1 & 74.71 & 7.40 & 19.215  \\
    xpln1 & 75.03 & 7.28 & 19.18     \\
    sway2 & 75.37 & 7.32 & 18.23     \\
    xpln2 & 75.03 & 7.27 & 19.24         
    \end{tabular}
  \end{center}
  \caption{Results for healthCloseIsses12mths0001-hard.csv}
  \label{tab:health-hard}
\end{table}

\begin{table}[]
  \begin{center}
    \begin{tabular}{lllll}
          & MRE- & ACC+ & PRED40+   \\
    \hline
    sway1 & 31.37 & -0.39 & 57.53   \\
    xpln1 & 35.96 & -0.46 & 53.85     \\
    sway2 & 26.03 & -0.46 & 62.11      \\
    xpln2 & 35.96 & -0.46 & 53.85          
    \end{tabular}
  \end{center}
  \caption{Results for healthCloseIsses12mths0011-easy.csv}
  \label{tab:health-easy}
\end{table}

% \begin{table}[]
%   \begin{center}
%     \begin{tabular}{lllll}
%           & PSNR-  &   Energy-    \\
%     \hline
%     sway1 & 44.48  &   1126.28   \\
%     xpln1 & 44.29  &   1621.32     \\
%     sway2 & 44.21  &   1612.90      \\
%     xpln2 & 44.46  &   1640.16          
%     \end{tabular}
%   \end{center}
%   \caption{Results for SSN.csv}
%   \label{tab:ssn}
% \end{table}


% \begin{table}[]
%   \begin{center}
%     \begin{tabular}{lllll}
%           & NUMBERITERATIONS-    \\
%     \hline
%     sway1 & 5.37   \\
%     xpln1 & 10.20     \\
%     sway2 & 6.27      \\
%     xpln2 & 9.37         
%     \end{tabular}
%   \end{center}
%   \caption{Results for SSM.csv}
%   \label{tab:ssm}
% \end{table}

\begin{table}[]
  \begin{center}
    \begin{tabular}{lllll}
          & LOC+   &    RISK- &   EFFORT-     \\
    \hline
    sway1 & 1004.94&    5.22   &  26372.00  \\
    xpln1 & 604.81  &   \myblue{4.02}  &  \myblue{17308.46}   \\
    sway2 & 1006.16  &  \myred{4.57}   &  \myred{24570.96}      \\
    xpln2 & 454.17   &  \myred{2.71}   &  \myred{13859.63}        
    \end{tabular}
  \end{center}
  \caption{Results for coc10000.csv}
  \label{tab:coc10000}
\end{table}


\begin{table}[]
  \begin{center}
  \begin{tabular}{lllll}
       & Kloc+  & Effort- & Defects- & Months- \\
  \hline
  10   & 70.52  & 329.06  & 2038.96  & 19.09   \\
  25   & 47.925 & 250.97  & 1743.88  & 17.87   \\
  50   & 64.64  & 311.22  & 2427     & 20.08   \\
  100  & 62.24  & 314.84  & 2325.34  & 19.14   \\
  200  & 94.33  & 602.88  & 3629.39  & 24.35   \\
  500  & 86.29  & 489.42  & 3323.69  & 22.12   \\
  1000 & 106.19 & 635.1   & 4117.55  & 25.47  
  \end{tabular}
\end{center}
\caption{Results for different sample size for nasa93dem.csv}
\label{tab:sample_nasa93dem}
\end{table}


\begin{table*}[h]
  \begin{center}
    \begin{tabular}{llllllllllll}
  \hline
  dataset & characteristic & all & & &   
                             sway1 & & & 
                             sway2 & & &  \\
  & & all & sway1 & sway2 & 
  sway2 & xpln1 & top & 
  xpln2 & top & \\
  \hline\hline
  auto2     & CityMPG+ & = & $\neq$ & $\neq$ &  
                         = & = & $\neq$ & 
                         = & $\neq$ & \\
            & HighwayMPG+ & = & $\neq$ & $\neq$ & 
                            = & = & $\neq$ & 
                            = & $\neq$ & \\
            & Weight- & = & $\neq$ & $\neq$ & 
                        = & = & $\neq$ & 
                        = & $\neq$ & \\
            & Class-  & = & $\neq$ & $\neq$ & 
                        $\neq$ & = & $\neq$ & 
                        = & $\neq$ & 
                        \\
  \hline
  auto93    & Lbs-  & = & $\neq$ & $\neq$ & 
                      = & =  & $\neq$ & 
                      $\neq$ & $\neq$ &  \\
            & Acc+  & = & $\neq$ & $\neq$ & 
                      = & $\neq$  & $\neq$ & 
                       $\neq$ & $\neq$ & 
                       \\
            & Mpg+  & = & $\neq$ & $\neq$ & 
                      = & $\neq$ & $\neq$ & 
                       $\neq$ & $\neq$ & 
                       \\
  \hline
  china     & N\_effort- & = & $\neq$ & $\neq$ &
                           = & $\neq$ & $\neq$ & 
                           = & $\neq$ & 
                           \\
  \hline
  coc1000   & LOC+   & = & = & = & 
                       = & = & $\neq$ & 
                       = & $\neq$ & 
                        \\
            & AEXP-  & = & $\neq$ & = & 
                       = & = & $\neq$ & 
                        = & $\neq$ & 
                        \\
            & PLEX-  & = & $\neq$ & = & 
                       = & = & $\neq$ & 
                       = & $\neq$ & 
                        \\
            & RISK-  & = & $\neq$ & = & 
                       = & =  & = & 
                       = & = & 
                        \\
            & EFFORT- & = & $\neq$ & $\neq$ & 
                        = & = & $\neq$ & 
                        = & $\neq$ & 
                        \\
  \hline
  coc10000   & Loc+   & = & = & = &
                        = & $\neq$ & n/a & 
                        $\neq$ & n/a & 
                        \\
             & Risk-  & = & $\neq$ & $\neq$ &
                        = & =  & n/a & 
                        $\neq$ & n/a & 
                         \\
             & Effort- &  = & $\neq$ & $\neq$ &
                          = & $\neq$ & n/a & 
                          $\neq$ & n/a & 
                           \\
  \hline
  health...0001-hard     & MRE-       & = & $\neq$ & $\neq$ & 
                                        = & = & n/a & 
                                        $\neq$ & n/a & 
                                        \\
                         & ACC+       & = & $\neq$ & $\neq$ & 
                                        = & $\neq$ & n/a & 
                                        = & n/a & 
                                        \\
                         & PRED40+    & = & $\neq$ & $\neq$ & 
                                        = & = & n/a & 
                                        = & n/a & 
                                        \\
  \hline
  health...0011-easy     & MRE-       & = & $\neq$ & $\neq$ & 
                                        = & = & n/a & 
                                        $\neq$ & n/a & 
                                        \\
                         & ACC+       & = & $\neq$ & $\neq$ & 
                                        = & $\neq$ & n/a & 
                                        = & n/a & 
                                        \\
                         & PRED40+    & = & $\neq$ & $\neq$ & 
                                        = & = & n/a & 
                                        $\neq$ & n/a & 
                                         \\
  \hline
  nasa93dem   & Kloc+         & = & = & = & 
                                = & = & $\neq$ & 
                                = & $\neq$ & 
                                \\
              & Effort-       & = & $\neq$ & = & 
                                = & = & $\neq$ & 
                                = & $\neq$ & 
                                \\
              & Defects-      & = & $\neq$ & = & 
                                = & = & $\neq$ & 
                                = & $\neq$ & 
                                \\
              & Months-       & = & = & = & 
                                = & = & $\neq$ & 
                                = & $\neq$ & 
                                \\
  \hline
  pom   & Cost-             & = & $\neq$ & $\neq$ & 
                              = & = & n/a & 
                              $\neq$ & n/a & 
                               \\
        & Completion+       & = & = & = & 
                              = & $\neq$ & n/a & 
                              $\neq$ & n/a & 
                               \\
        & Idle-             & = & $\neq$ & = & 
                              = & $\neq$ & n/a & 
                              $\neq$ & n/a & 
                              \\
  \hline
  SSM  & NUMBERITERATIONS- &  = & $\neq$ & $\neq$ & 
                              = & $\neq$ & n/a & 
                              $\neq$ & n/a & 
                              \\
  \hline
  SSN     & PSNR-       & = & = & = & 
                          = & = & n/a & 
                          = & n/a & 
                          \\
          & Energy-     & = & $\neq$ & = &
                          = & $\neq$ & n/a & 
                          = & n/a & 
                          \\      
  \end{tabular}

  \end{center}
  \caption{Results for effect size test and significance test}
\label{tab:tests}
\end{table*}


\section{Discussion}
\label{sec:discussion}

\section{Bonus: Requirements Study}
\label{sec:bonus_requirements}

For our requirements study, we collected five reqgrids with a common
theme of TV shows. We asked the participants to come up with TV shows
and characteristics and rate the shows based on them. An example of the
output for clustering the TV shows can be seen in listing
\ref{lst:grid}. The outputs of each grid can be found in the GitHub
repository. A common characteristic
many people came up with was whether the show was easy to watch or full
of suspense. From the results, we observe that this was often also the
most significant feature that separated shows. We could infer that the
story and tone of the show could have a great influence on viewers'
rating of the show. Other less significant feature included whether the
show was realistic, episode length, connection to previous seasons, etc.
We might conclude from the results that these characteristics have less
of an effect on how the audience rates the show.

\begin{minipage}{\linewidth}
  \begin{lstlisting}[
    basicstyle=\small,
    caption={example of repgrid output},
    label={lst:grid}
  ]
    90
    |..69
    |..|..41
    |..|..|..Schitts Creek
    |..|..|..New Girl
    |..|..65
    |..|..|..Narcos
    |..|..|..Ted Lasso
    |..84
    |..|..61
    |..|..|..The Last of Us
    |..|..|..Beef
    |..|..84
    |..|..|..Loki
    |..|..|..84
    |..|..|..|..Greys Anatomy
    |..|..|..|..Moonknight
    80
    |..50
    |..|..Easy to watch:Full of Suspense
    |..|..60
    |..|..|..not much diversity:diverse
    |..|..|..Family Friendly:violent
    |..72
    |..|..short:long total length
    |..|..42
    |..|..|..poor:good visual effects
    |..|..|..normal:good looking characters
  \end{lstlisting}
\end{minipage}


\section{Bonus: February Study}
\label{sec:bonus_february}

For our february study, we take a look back at our results for xpln. As
xpln picks a rule to best distinguish between the best and rest data
returned by sway. We could make use of results and rules from a previous
study to help in future studies. Xpln performs better than sway in
certain datasets, which we highlighted in blue in the tables. 
If we were to minimize risk in the dataset
\textit{coc10000}, we could filter data first according to the rule
given by xpln. In one of our runs of experiments on \textit{coc10000},
the rule \verb|{data ['n']}| chosen by xpln yielded better results than
sway. For future studies, we could collect a set of rules that xpln
performs better on, and preprocess the data using those rules to get
better results.

\section{Bonus: Ablation Study}
\label{sec:bonus_ablation}

\section{Bonus: HPO study}
\label{sec:bonus_hpo}




\bibliographystyle{abbrv}
\bibliography{main}




% that's all folks
\end{document}


